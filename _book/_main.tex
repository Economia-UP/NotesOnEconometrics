% Options for packages loaded elsewhere
\PassOptionsToPackage{unicode}{hyperref}
\PassOptionsToPackage{hyphens}{url}
%
\documentclass[
]{book}
\usepackage{amsmath,amssymb}
\usepackage{iftex}
\ifPDFTeX
  \usepackage[T1]{fontenc}
  \usepackage[utf8]{inputenc}
  \usepackage{textcomp} % provide euro and other symbols
\else % if luatex or xetex
  \usepackage{unicode-math} % this also loads fontspec
  \defaultfontfeatures{Scale=MatchLowercase}
  \defaultfontfeatures[\rmfamily]{Ligatures=TeX,Scale=1}
\fi
\usepackage{lmodern}
\ifPDFTeX\else
  % xetex/luatex font selection
\fi
% Use upquote if available, for straight quotes in verbatim environments
\IfFileExists{upquote.sty}{\usepackage{upquote}}{}
\IfFileExists{microtype.sty}{% use microtype if available
  \usepackage[]{microtype}
  \UseMicrotypeSet[protrusion]{basicmath} % disable protrusion for tt fonts
}{}
\makeatletter
\@ifundefined{KOMAClassName}{% if non-KOMA class
  \IfFileExists{parskip.sty}{%
    \usepackage{parskip}
  }{% else
    \setlength{\parindent}{0pt}
    \setlength{\parskip}{6pt plus 2pt minus 1pt}}
}{% if KOMA class
  \KOMAoptions{parskip=half}}
\makeatother
\usepackage{xcolor}
\usepackage{color}
\usepackage{fancyvrb}
\newcommand{\VerbBar}{|}
\newcommand{\VERB}{\Verb[commandchars=\\\{\}]}
\DefineVerbatimEnvironment{Highlighting}{Verbatim}{commandchars=\\\{\}}
% Add ',fontsize=\small' for more characters per line
\usepackage{framed}
\definecolor{shadecolor}{RGB}{248,248,248}
\newenvironment{Shaded}{\begin{snugshade}}{\end{snugshade}}
\newcommand{\AlertTok}[1]{\textcolor[rgb]{0.94,0.16,0.16}{#1}}
\newcommand{\AnnotationTok}[1]{\textcolor[rgb]{0.56,0.35,0.01}{\textbf{\textit{#1}}}}
\newcommand{\AttributeTok}[1]{\textcolor[rgb]{0.13,0.29,0.53}{#1}}
\newcommand{\BaseNTok}[1]{\textcolor[rgb]{0.00,0.00,0.81}{#1}}
\newcommand{\BuiltInTok}[1]{#1}
\newcommand{\CharTok}[1]{\textcolor[rgb]{0.31,0.60,0.02}{#1}}
\newcommand{\CommentTok}[1]{\textcolor[rgb]{0.56,0.35,0.01}{\textit{#1}}}
\newcommand{\CommentVarTok}[1]{\textcolor[rgb]{0.56,0.35,0.01}{\textbf{\textit{#1}}}}
\newcommand{\ConstantTok}[1]{\textcolor[rgb]{0.56,0.35,0.01}{#1}}
\newcommand{\ControlFlowTok}[1]{\textcolor[rgb]{0.13,0.29,0.53}{\textbf{#1}}}
\newcommand{\DataTypeTok}[1]{\textcolor[rgb]{0.13,0.29,0.53}{#1}}
\newcommand{\DecValTok}[1]{\textcolor[rgb]{0.00,0.00,0.81}{#1}}
\newcommand{\DocumentationTok}[1]{\textcolor[rgb]{0.56,0.35,0.01}{\textbf{\textit{#1}}}}
\newcommand{\ErrorTok}[1]{\textcolor[rgb]{0.64,0.00,0.00}{\textbf{#1}}}
\newcommand{\ExtensionTok}[1]{#1}
\newcommand{\FloatTok}[1]{\textcolor[rgb]{0.00,0.00,0.81}{#1}}
\newcommand{\FunctionTok}[1]{\textcolor[rgb]{0.13,0.29,0.53}{\textbf{#1}}}
\newcommand{\ImportTok}[1]{#1}
\newcommand{\InformationTok}[1]{\textcolor[rgb]{0.56,0.35,0.01}{\textbf{\textit{#1}}}}
\newcommand{\KeywordTok}[1]{\textcolor[rgb]{0.13,0.29,0.53}{\textbf{#1}}}
\newcommand{\NormalTok}[1]{#1}
\newcommand{\OperatorTok}[1]{\textcolor[rgb]{0.81,0.36,0.00}{\textbf{#1}}}
\newcommand{\OtherTok}[1]{\textcolor[rgb]{0.56,0.35,0.01}{#1}}
\newcommand{\PreprocessorTok}[1]{\textcolor[rgb]{0.56,0.35,0.01}{\textit{#1}}}
\newcommand{\RegionMarkerTok}[1]{#1}
\newcommand{\SpecialCharTok}[1]{\textcolor[rgb]{0.81,0.36,0.00}{\textbf{#1}}}
\newcommand{\SpecialStringTok}[1]{\textcolor[rgb]{0.31,0.60,0.02}{#1}}
\newcommand{\StringTok}[1]{\textcolor[rgb]{0.31,0.60,0.02}{#1}}
\newcommand{\VariableTok}[1]{\textcolor[rgb]{0.00,0.00,0.00}{#1}}
\newcommand{\VerbatimStringTok}[1]{\textcolor[rgb]{0.31,0.60,0.02}{#1}}
\newcommand{\WarningTok}[1]{\textcolor[rgb]{0.56,0.35,0.01}{\textbf{\textit{#1}}}}
\usepackage{longtable,booktabs,array}
\usepackage{calc} % for calculating minipage widths
% Correct order of tables after \paragraph or \subparagraph
\usepackage{etoolbox}
\makeatletter
\patchcmd\longtable{\par}{\if@noskipsec\mbox{}\fi\par}{}{}
\makeatother
% Allow footnotes in longtable head/foot
\IfFileExists{footnotehyper.sty}{\usepackage{footnotehyper}}{\usepackage{footnote}}
\makesavenoteenv{longtable}
\usepackage{graphicx}
\makeatletter
\def\maxwidth{\ifdim\Gin@nat@width>\linewidth\linewidth\else\Gin@nat@width\fi}
\def\maxheight{\ifdim\Gin@nat@height>\textheight\textheight\else\Gin@nat@height\fi}
\makeatother
% Scale images if necessary, so that they will not overflow the page
% margins by default, and it is still possible to overwrite the defaults
% using explicit options in \includegraphics[width, height, ...]{}
\setkeys{Gin}{width=\maxwidth,height=\maxheight,keepaspectratio}
% Set default figure placement to htbp
\makeatletter
\def\fps@figure{htbp}
\makeatother
\setlength{\emergencystretch}{3em} % prevent overfull lines
\providecommand{\tightlist}{%
  \setlength{\itemsep}{0pt}\setlength{\parskip}{0pt}}
\setcounter{secnumdepth}{5}
\usepackage{booktabs}
\ifLuaTeX
  \usepackage{selnolig}  % disable illegal ligatures
\fi
\usepackage[]{natbib}
\bibliographystyle{plainnat}
\IfFileExists{bookmark.sty}{\usepackage{bookmark}}{\usepackage{hyperref}}
\IfFileExists{xurl.sty}{\usepackage{xurl}}{} % add URL line breaks if available
\urlstyle{same}
\hypersetup{
  pdftitle={A Minimal Book Example},
  pdfauthor={John Doe},
  hidelinks,
  pdfcreator={LaTeX via pandoc}}

\title{A Minimal Book Example}
\author{John Doe}
\date{2025-02-16}

\begin{document}
\maketitle

{
\setcounter{tocdepth}{1}
\tableofcontents
}
\hypertarget{about}{%
\chapter{About}\label{about}}

This is a \emph{sample} book written in \textbf{Markdown}. You can use anything that Pandoc's Markdown supports; for example, a math equation \(a^2 + b^2 = c^2\).

\hypertarget{usage}{%
\section{Usage}\label{usage}}

Each \textbf{bookdown} chapter is an .Rmd file, and each .Rmd file can contain one (and only one) chapter. A chapter \emph{must} start with a first-level heading: \texttt{\#\ A\ good\ chapter}, and can contain one (and only one) first-level heading.

Use second-level and higher headings within chapters like: \texttt{\#\#\ A\ short\ section} or \texttt{\#\#\#\ An\ even\ shorter\ section}.

The \texttt{index.Rmd} file is required, and is also your first book chapter. It will be the homepage when you render the book.

\hypertarget{render-book}{%
\section{Render book}\label{render-book}}

You can render the HTML version of this example book without changing anything:

\begin{enumerate}
\def\labelenumi{\arabic{enumi}.}
\item
  Find the \textbf{Build} pane in the RStudio IDE, and
\item
  Click on \textbf{Build Book}, then select your output format, or select ``All formats'' if you'd like to use multiple formats from the same book source files.
\end{enumerate}

Or build the book from the R console:

\begin{Shaded}
\begin{Highlighting}[]
\NormalTok{bookdown}\SpecialCharTok{::}\FunctionTok{render\_book}\NormalTok{()}
\end{Highlighting}
\end{Shaded}

To render this example to PDF as a \texttt{bookdown::pdf\_book}, you'll need to install XeLaTeX. You are recommended to install TinyTeX (which includes XeLaTeX): \url{https://yihui.org/tinytex/}.

\hypertarget{preview-book}{%
\section{Preview book}\label{preview-book}}

As you work, you may start a local server to live preview this HTML book. This preview will update as you edit the book when you save individual .Rmd files. You can start the server in a work session by using the RStudio add-in ``Preview book'', or from the R console:

\begin{Shaded}
\begin{Highlighting}[]
\NormalTok{bookdown}\SpecialCharTok{::}\FunctionTok{serve\_book}\NormalTok{()}
\end{Highlighting}
\end{Shaded}

\hypertarget{introducciuxf3n}{%
\chapter{Introducción}\label{introducciuxf3n}}

\hypertarget{anuxe1lisis-de-regresiuxf3n}{%
\chapter{Análisis de regresión}\label{anuxe1lisis-de-regresiuxf3n}}

Econometría: Medición económica.

\hypertarget{metodologuxeda-cluxe1sica}{%
\section{Metodología clásica}\label{metodologuxeda-cluxe1sica}}

\begin{enumerate}
\def\labelenumi{\arabic{enumi}.}
\tightlist
\item
  Planteamiento de teoría (hipótesis)\\
\item
  Especificación del modelo matemático\\
\item
  Especificación del modelo econométrico\\
\item
  Obtención de datos\\
\item
  Estimación de parámetros del modelo\\
\item
  Pruebas de hipótesis\\
\item
  Pronóstico (predicción)\\
\item
  Modelo para fines de control/política
\end{enumerate}

Ej. Función consumo keynesiana:

\[
c = \alpha + \beta y \quad \forall \ 0< \beta <1
\]

Las relaciones entre variables económicas son \textbf{inexactas}, dada la injerencia de otras variables:

\[ 
\underset{\text{Modelo econométrico} }{c = \alpha + \beta y + u}
\]

\[ 
\underset{ \text{Variable aleatoria con propiedades probabilísticas} }{ u = \text{Error (perturbación estocástica)} } 
\]

\(u\) incluye todos los factores que afectan \textbf{consumo} pero no están en la ecuación: tamaño de familia, edades, etc.

El modelo requiere ser estimado: obtener valores \(\alpha\) y \(\beta\) a partir de \textbf{datos.}

Ej. Gasto en consumo personal

\begin{itemize}
\tightlist
\item
  Regresión: técnica estadística para el estudio de una \textbf{variable dependiente} que está en función de una o más \textbf{variables independientes.}
\item
  Usando los datos de consumo y PIB de BUA se obtiene:
\end{itemize}

\[ 
\hat c = -231.8 + 0.7194 y
\]

Donde \(\alpha = -231.8, \beta = 0.7194, \hat c = \text{Consumo (estimado)}, y = \text{PIB}\).

La interpretación consiste en que un incremento de tasa en el ingreso incrementa (en promedio) el consumo en 0.72 USD.

Se debe probar si los valores estimados:

\begin{enumerate}
\def\labelenumi{\arabic{enumi}.}
\tightlist
\item
  Son estadísticamente significativos \((\alpha, \beta \neq 0)\)\\
\item
  Confirman la teoría (hipótesis) que está siendo probada \((0<\beta<1)\)
\end{enumerate}

Si el modelo confirma la teoría (hipótesis), se pueden \textbf{pronosticar} valores futuros de la variable dependiente.

Ej. Suponer un PIB esperado para 1995 de \(6,000\) mmd, ¿cuál es el pronóstico de consumo?

\[
\hat c = -231.8 + 0.7194(6,000) = 4,084.6 
\]

Suponer que el gobierno considera que un gasto de \$4,000 mmd mantendrá la tasa de desempleo en 6.5\%. ¿Cuál nivel de ingreso garantizará esta meta de consumo?

\[
\begin{aligned} 
\hat c &= -231.8 + 0.7194y \\
4,000 &= -231.8 + 0.7194y \\ 
0.7194y &= 4,000 + 231.8 \\ 
y &= \frac{4,231.8}{0.7194} \\ 
y^* &= 5,882.40 
\end{aligned}
\]

Un modelo estimado puede ser usado para fines de control o de política económica (fiscal y monetaria).

\begin{center}\rule{0.5\linewidth}{0.5pt}\end{center}

\hypertarget{probabilidad-condicional}{%
\section{Probabilidad condicional}\label{probabilidad-condicional}}

Suponer un país con una población de 60 familias. Se estudia el gasto en consumo familiar semanal \((y)\) y el ingreso familiar semanal \((x)\).

Se presenta la distribución del gasto en consumo \((y)\) correspondiente a un ingreso fijo \((x)\): la distribución condicional de \(y\) dada \(x\).

Para \(P(y|x = 80)\):

\[ 
\begin{aligned} 
P(y=55|x=80) &= \frac{1}{5} \\ 
P(y=60|x=80) &= \frac{1}{5} \\ 
P(y=65|x=80) &= \frac{1}{5} \\ 
P(y=70|x=80) &= \frac{1}{5} \\ 
P(y=75|x=80) &= \frac{1}{5} 
\end{aligned} 
\]

Para cada distribución de probabilidad condicional de \(y\), calculamos su \textbf{media (media condicional):}

\[ 
E(y|x=80) = 55\left( \frac{1}{5} \right) + 60\left( \frac{1}{5} \right) + 65\left( \frac{1}{5} \right) + 70\left( \frac{1}{5} \right) + 75\left( \frac{1}{5} \right) = 65
\]

\[
\mu_{y|x=100} = \frac{\sum_{y}y_{i}}{n} = \frac{462}{6} = 77
\]

\begin{center}\rule{0.5\linewidth}{0.5pt}\end{center}

\hypertarget{funciuxf3n-de-regresiuxf3n-poblacional}{%
\section{Función de regresión poblacional}\label{funciuxf3n-de-regresiuxf3n-poblacional}}

Lugar geométrico de las medias condicionales de la variable dependiente para valores fijos de la variable independiente.

Se puede deducir que: \(E(y|x_i) = f(x_i)\).

\[
E(y|x_i) \rightarrow \text{Función de Regresión Poblacional (FRP)}
\]

Forma funcional de la FRP:

\[
E(y|x_i) = \alpha + \beta x_i \rightarrow \underset{y = ax+b}{\text{Ecuación de recta}}
\]

\[
\begin{array}{ccc} 
\alpha, \beta & \rightarrow & \text{Coeficientes de regresión} \\
\alpha & \rightarrow & \text{Intercepto} \\
\beta & \rightarrow & \text{Coeficiente de la pendiente}
\end{array}
\]

\textbf{Objetivo:} Estimar \(\alpha\) y \(\beta\) con base en observaciones de \(x\) y \(y\).\\
Esta desviación de un \(y_i\) alrededor de su valor esperado se expresa como:

\[
\begin{aligned}
u_i &= y_i - E(y|x_i) \\
y_i &= E(y|x_i) + u_i
\end{aligned}
\]

Donde \(u_i\) es el término de error estocástico, que representa todas las variables omitidas que puedan afectar \(y\), pero no están incluidas en el modelo de regresión.

\[
\begin{aligned}
y_i = \alpha + \beta x_i + u_i \quad \therefore \quad E(y|x_i) = \alpha + \beta x_i \\
\therefore \quad y_i = \alpha + \beta x_i + u_i \rightarrow \text{Ecuación de regresión}
\end{aligned}
\]

Adicionalmente, si se toma \(E(\cdot)\) a \(y_i\), tenemos que:
\[
\begin{aligned}
y_i &= E(y|x_i) + u_i \\
E(y|x_i) &= E[E(y|x_i)] + E(u_i|x_i) \\
&= E(y|x_i) + E(u_i|x_i) \quad \therefore \\
E(u_i|x_i) &= E(y|x_i) - E(y|x_i) \quad \therefore \\
E(u_i|x_i) &= 0
\end{aligned}
\]

\begin{center}\rule{0.5\linewidth}{0.5pt}\end{center}

\hypertarget{funciuxf3n-de-regresiuxf3n-muestral}{%
\section{Función de regresión muestral}\label{funciuxf3n-de-regresiuxf3n-muestral}}

A diferencia del caso anterior, en realidad se trabaja con muestras. Se estima la FRP con base en información \textbf{muestral.}
Suponer que se extrae una muestra d ela población de 60 familias donde se estudia el gasto en consumo y el ingreso familiar.
Muestra aleatoria de 10 observaciones.

{[}IMAGEN{]}

La función de regresión muestral (FRM) puede escribirse como:
\[
\hat y_i = \hat \alpha + \hat \beta x_i 
\]
Donde:
- \(\hat y_i \equiv\) \(y\) estimada
- \(\hat \alpha_i \equiv\) estimador de \(\alpha\)
- \(\hat \beta_i \equiv\) estimador de \(\beta\)

Se estiman los parámetros poblacionales a partir de información muestral.

\[
\begin{array}{ccc}
\text{Estimador} & \rightarrow & \text{Fórmula} \\
\text{Estimado} & \rightarrow & \text{Valor numérico}
\end{array}
\]

FRM (forma estocástica): \(y_i = \hat \alpha + \hat \beta x_i + \hat u_i\). Por lo tanto, \(\hat u_i\) es el residual muestral (estimado de \(u_i\)).

\begin{center}\rule{0.5\linewidth}{0.5pt}\end{center}

\hypertarget{objetivo-del-anuxe1lisis-de-regresiuxf3n}{%
\section{Objetivo del análisis de regresión}\label{objetivo-del-anuxe1lisis-de-regresiuxf3n}}

Estimar FRP, \(y_i = \alpha + \beta x_i + u_i\), a partir de estimar FRM, \(y_i = \hat \alpha + \hat \beta x_i + \hat u_i\).

{[}IMAGEN{]}

\hypertarget{conceptos-estaduxedsticos}{%
\section{Conceptos estadísticos}\label{conceptos-estaduxedsticos}}

\hypertarget{sumatoria-y-multiplicatoria}{%
\subsection{Sumatoria y multiplicatoria}\label{sumatoria-y-multiplicatoria}}

\[
\sum_{i=1}^n x_i = x_i + x_2 + \ldots + x_n
\]
Propiedades del operador de sumatoria \((\sum)\):

\begin{enumerate}
\def\labelenumi{\arabic{enumi}.}
\tightlist
\item
  \(\sum_{i=k}^n = nk \quad \therefore \quad k \equiv \text{Constante}\).
\item
  \(\sum_{i=1}^n k x_i = k\sum_{i=1}^nx_i \quad \therefore \quad k \equiv \text{Constante}\).
\item
  \(\sum_{i=1}^n (a+bx_i) = \sum_{i=1}^n a + \sum_{i=1}^n bx_i = na + b \sum_{i=1}^n x_i \quad \forall \quad a,b \equiv \text{Constante}\).
\item
  \(\sum_{i=1}^n (x_i + y_i) = \sum_{i=1}^nx_i + \sum_{i=1}^n y_i\).
\end{enumerate}

Se pueden tener sumatorias múltiples:
\[
\begin{aligned}
\sum_{i=1}^n\sum_{j=1}^m &= \sum_{i=1}^n (x_{i1} + x_{i2} + \ldots + x_{im}) \\
&= (x_{11} + x_{12} + \ldots + x_{1m}) + (x_{21} + x_{22} + \ldots + x_{2m})  + \ldots + (x_{n1} + x_{n2} + \ldots + x_{nm})
\end{aligned}
\]
Propiedades del operador de sumatoria \((\sum \sum)\):

\begin{enumerate}
\def\labelenumi{\arabic{enumi}.}
\tightlist
\item
  \(\sum_{i=1}^n\sum_{j=1}^m x_{ij} = \sum_{j=1}^m\sum_{i=1}^n \quad \rightarrow \quad \text{Intercambiable}\)
\item
  \(\sum_{i=1}^n\sum_{j=1}^m x_i y_j = \sum_{i=1}^n x_i \sum_{j=1}^m y_j\)
\item
  \(\sum_{i=1}^n\sum_{j=1}^m (x_{ij} + y_{ij}) = \sum_{i=1}^n\sum_{j=1}^m x_{ij} + \sum_{i=1}^n\sum_{j=1}^m y_{ij}\)
\end{enumerate}

Adicionalmente, el operador multiplicatoria:
\[
\Pi_{i=1}^n x_i = x_1\cdot x_2 \cdot \ldots \cdot x_n
\]

\hypertarget{valor-esperado}{%
\subsection{Valor esperado}\label{valor-esperado}}

El valor esperado de una variable aleatoria discreta se define como:
\[
E(x) = \sum_x xf(x)
\]
\[
\begin{array}{ccc}
\therefore \quad x & \rightarrow & \text{Valores de la variable aleatoria discreta} \\
f(x) & \rightarrow & \text{FDP (discreta) de x}
\end{array}
\]

En términos poblacionales:

\[
E(x) = \mu_x \ quad \ rightarrow \quad \text{La media de la variable aleatoria discreta}
\]

Propiedades del valor esperado \(E(\cdot)\):

\begin{enumerate}
\def\labelenumi{\arabic{enumi}.}
\tightlist
\item
  \(E(b) = b \quad \therefore \quad b \equiv \text{Constante}\)
\item
  \(E(ax + b) = E(ax) + E(b) = aE(x) + b \quad \forall \quad a,b \text{Constantes}\)
\item
  \(E(xy) = E(x)E(y) \quad \forall \quad x,y \quad \rightarrow \quad \text{Variables aleatorias independientes}\)
\end{enumerate}

\hypertarget{varianza}{%
\subsection{Varianza}\label{varianza}}

  \bibliography{book.bib,packages.bib}

\end{document}
