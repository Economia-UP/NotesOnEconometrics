% Options for packages loaded elsewhere
\PassOptionsToPackage{unicode}{hyperref}
\PassOptionsToPackage{hyphens}{url}
%
\documentclass[
]{book}
\usepackage{amsmath,amssymb}
\usepackage{iftex}
\ifPDFTeX
  \usepackage[T1]{fontenc}
  \usepackage[utf8]{inputenc}
  \usepackage{textcomp} % provide euro and other symbols
\else % if luatex or xetex
  \usepackage{unicode-math} % this also loads fontspec
  \defaultfontfeatures{Scale=MatchLowercase}
  \defaultfontfeatures[\rmfamily]{Ligatures=TeX,Scale=1}
\fi
\usepackage{lmodern}
\ifPDFTeX\else
  % xetex/luatex font selection
\fi
% Use upquote if available, for straight quotes in verbatim environments
\IfFileExists{upquote.sty}{\usepackage{upquote}}{}
\IfFileExists{microtype.sty}{% use microtype if available
  \usepackage[]{microtype}
  \UseMicrotypeSet[protrusion]{basicmath} % disable protrusion for tt fonts
}{}
\makeatletter
\@ifundefined{KOMAClassName}{% if non-KOMA class
  \IfFileExists{parskip.sty}{%
    \usepackage{parskip}
  }{% else
    \setlength{\parindent}{0pt}
    \setlength{\parskip}{6pt plus 2pt minus 1pt}}
}{% if KOMA class
  \KOMAoptions{parskip=half}}
\makeatother
\usepackage{xcolor}
\usepackage{longtable,booktabs,array}
\usepackage{calc} % for calculating minipage widths
% Correct order of tables after \paragraph or \subparagraph
\usepackage{etoolbox}
\makeatletter
\patchcmd\longtable{\par}{\if@noskipsec\mbox{}\fi\par}{}{}
\makeatother
% Allow footnotes in longtable head/foot
\IfFileExists{footnotehyper.sty}{\usepackage{footnotehyper}}{\usepackage{footnote}}
\makesavenoteenv{longtable}
\usepackage{graphicx}
\makeatletter
\def\maxwidth{\ifdim\Gin@nat@width>\linewidth\linewidth\else\Gin@nat@width\fi}
\def\maxheight{\ifdim\Gin@nat@height>\textheight\textheight\else\Gin@nat@height\fi}
\makeatother
% Scale images if necessary, so that they will not overflow the page
% margins by default, and it is still possible to overwrite the defaults
% using explicit options in \includegraphics[width, height, ...]{}
\setkeys{Gin}{width=\maxwidth,height=\maxheight,keepaspectratio}
% Set default figure placement to htbp
\makeatletter
\def\fps@figure{htbp}
\makeatother
\setlength{\emergencystretch}{3em} % prevent overfull lines
\providecommand{\tightlist}{%
  \setlength{\itemsep}{0pt}\setlength{\parskip}{0pt}}
\setcounter{secnumdepth}{5}
\usepackage{booktabs}
\usepackage{cancel}
\ifLuaTeX
  \usepackage{selnolig}  % disable illegal ligatures
\fi
\usepackage[]{natbib}
\bibliographystyle{plainnat}
\IfFileExists{bookmark.sty}{\usepackage{bookmark}}{\usepackage{hyperref}}
\IfFileExists{xurl.sty}{\usepackage{xurl}}{} % add URL line breaks if available
\urlstyle{same}
\hypersetup{
  pdftitle={\#NotesOnEconometrics},
  pdfauthor={Economía UP},
  hidelinks,
  pdfcreator={LaTeX via pandoc}}

\title{\#NotesOnEconometrics}
\author{Economía UP}
\date{}

\begin{document}
\maketitle

{
\setcounter{tocdepth}{1}
\tableofcontents
}
\hypertarget{bienvenido}{%
\chapter{Bienvenido}\label{bienvenido}}

\textbf{\#NotesonEconometrics} es una iniciativa creada por alumnos de la Licenciatura en Economía de la Universidad Panamericana, con el objetivo de ofrecer recursos académicos accesibles y de alta calidad sobre econometría. Este proyecto forma parte de Economía UP, un esfuerzo estudiantil dedicado a fomentar el aprendizaje y el análisis económico de manera rigurosa y práctica, poniendo a disposición de los estudiantes herramientas que faciliten el estudio y la aplicación de conceptos econométricos fundamentales y avanzados.

En este sitio encontrarás contenido sintetizado de algunos de los libros más importantes en el campo de la econometría, como \emph{Introductory Econometrics} de Jeffrey Wooldridge, \emph{Basic Econometrics} de Damodar Gujarati e \emph{Introduction to Econometrics} de G.S. Maddala. Estas obras han sido seleccionadas por su relevancia académica y su capacidad para explicar conceptos complejos de manera accesible. Además, el proyecto incorpora notas y recursos de cursos impartidos por profesores de la Universidad Panamericana, como Javier Alcántar Toledo y Eugenio Gómez Alatorre, cuyas enseñanzas han sido fundamentales para la formación en econometría de muchos estudiantes.

El contenido de \textbf{\#NotesOnEconometrics} está diseñado para ayudar a los estudiantes a comprender y aplicar conceptos clave de econometría. A través de explicaciones claras, ejemplos prácticos y tutoriales de R y Python, buscamos no solo simplificar el aprendizaje de la teoría econométrica, sino también facilitar su aplicación en problemas del mundo real. Esto incluye desde los fundamentos de la regresión lineal hasta temas más avanzados, proporcionando herramientas que permitan a los usuarios desarrollar habilidades analíticas rigurosas y aplicarlas en investigaciones académicas o en el análisis de datos económicos.

Este proyecto está dirigido a estudiantes de economía, investigadores y cualquier persona interesada en aprender econometría de manera autodidacta o complementar sus estudios formales. Si estás cursando materias de econometría en la universidad o deseas fortalecer tus habilidades analíticas para enfrentar retos en el ámbito profesional, este sitio ha sido pensado para ti. La misión de \textbf{\#NotesOnEconometrics} es hacer de la econometría un tema accesible y comprensible, apoyando a los estudiantes en su desarrollo académico y profesional a través de recursos cuidadosamente seleccionados y organizados.

\hypertarget{anuxe1lisis-de-regresiuxf3n}{%
\chapter{Análisis de regresión}\label{anuxe1lisis-de-regresiuxf3n}}

Econometría: Medición económica.

\hypertarget{metodologuxeda-cluxe1sica}{%
\section{Metodología clásica}\label{metodologuxeda-cluxe1sica}}

\begin{enumerate}
\def\labelenumi{\arabic{enumi}.}
\tightlist
\item
  Planteamiento de teoría (hipótesis)\\
\item
  Especificación del modelo matemático\\
\item
  Especificación del modelo econométrico\\
\item
  Obtención de datos\\
\item
  Estimación de parámetros del modelo\\
\item
  Pruebas de hipótesis\\
\item
  Pronóstico (predicción)\\
\item
  Modelo para fines de control/política
\end{enumerate}

Ej. Función consumo keynesiana:

\[
c = \alpha + \beta y \quad \forall \ 0< \beta <1
\]

Las relaciones entre variables económicas son \textbf{inexactas}, dada la injerencia de otras variables:

\[ 
\underset{\text{Modelo econométrico} }{c = \alpha + \beta y + u}
\]

\[ 
\underset{ \text{Variable aleatoria con propiedades probabilísticas} }{ u = \text{Error (perturbación estocástica)} } 
\]

\(u\) incluye todos los factores que afectan \textbf{consumo} pero no están en la ecuación: tamaño de familia, edades, etc.

El modelo requiere ser estimado: obtener valores \(\alpha\) y \(\beta\) a partir de \textbf{datos.}

Ej. Gasto en consumo personal

\begin{itemize}
\tightlist
\item
  Regresión: técnica estadística para el estudio de una \textbf{variable dependiente} que está en función de una o más \textbf{variables independientes.}
\item
  Usando los datos de consumo y PIB de BUA se obtiene:
\end{itemize}

\[ 
\hat c = -231.8 + 0.7194 y
\]

Donde \(\alpha = -231.8, \beta = 0.7194, \hat c = \text{Consumo (estimado)}, y = \text{PIB}\).

La interpretación consiste en que un incremento de tasa en el ingreso incrementa (en promedio) el consumo en 0.72 USD.

Se debe probar si los valores estimados:

\begin{enumerate}
\def\labelenumi{\arabic{enumi}.}
\tightlist
\item
  Son estadísticamente significativos \((\alpha, \beta \neq 0)\)\\
\item
  Confirman la teoría (hipótesis) que está siendo probada \((0<\beta<1)\)
\end{enumerate}

Si el modelo confirma la teoría (hipótesis), se pueden \textbf{pronosticar} valores futuros de la variable dependiente.

Ej. Suponer un PIB esperado para 1995 de \(6,000\) mmd, ¿cuál es el pronóstico de consumo?

\[
\hat c = -231.8 + 0.7194(6,000) = 4,084.6 
\]

Suponer que el gobierno considera que un gasto de \$4,000 mmd mantendrá la tasa de desempleo en 6.5\%. ¿Cuál nivel de ingreso garantizará esta meta de consumo?

\[
\begin{aligned} 
\hat c &= -231.8 + 0.7194y \\
4,000 &= -231.8 + 0.7194y \\ 
0.7194y &= 4,000 + 231.8 \\ 
y &= \frac{4,231.8}{0.7194} \\ 
y^* &= 5,882.40 
\end{aligned}
\]

Un modelo estimado puede ser usado para fines de control o de política económica (fiscal y monetaria).

\begin{center}\rule{0.5\linewidth}{0.5pt}\end{center}

\hypertarget{probabilidad-condicional}{%
\section{Probabilidad condicional}\label{probabilidad-condicional}}

Suponer un país con una población de 60 familias. Se estudia el gasto en consumo familiar semanal \((y)\) y el ingreso familiar semanal \((x)\).

Se presenta la distribución del gasto en consumo \((y)\) correspondiente a un ingreso fijo \((x)\): la distribución condicional de \(y\) dada \(x\).

Para \(P(y|x = 80)\):

\[ 
\begin{aligned} 
P(y=55|x=80) &= \frac{1}{5} \\ 
P(y=60|x=80) &= \frac{1}{5} \\ 
P(y=65|x=80) &= \frac{1}{5} \\ 
P(y=70|x=80) &= \frac{1}{5} \\ 
P(y=75|x=80) &= \frac{1}{5} 
\end{aligned} 
\]

Para cada distribución de probabilidad condicional de \(y\), calculamos su \textbf{media (media condicional):}

\[ 
E(y|x=80) = 55\left( \frac{1}{5} \right) + 60\left( \frac{1}{5} \right) + 65\left( \frac{1}{5} \right) + 70\left( \frac{1}{5} \right) + 75\left( \frac{1}{5} \right) = 65
\]

\[
\mu_{y|x=100} = \frac{\sum_{y}y_{i}}{n} = \frac{462}{6} = 77
\]

\begin{center}\rule{0.5\linewidth}{0.5pt}\end{center}

\hypertarget{funciuxf3n-de-regresiuxf3n-poblacional}{%
\section{Función de regresión poblacional}\label{funciuxf3n-de-regresiuxf3n-poblacional}}

Lugar geométrico de las medias condicionales de la variable dependiente para valores fijos de la variable independiente.

Se puede deducir que: \(E(y|x_i) = f(x_i)\).

\[
E(y|x_i) \rightarrow \text{Función de Regresión Poblacional (FRP)}
\]

Forma funcional de la FRP:

\[
E(y|x_i) = \alpha + \beta x_i \rightarrow \underset{y = ax+b}{\text{Ecuación de recta}}
\]

\[
\begin{array}{ccc} 
\alpha, \beta & \rightarrow & \text{Coeficientes de regresión} \\
\alpha & \rightarrow & \text{Intercepto} \\
\beta & \rightarrow & \text{Coeficiente de la pendiente}
\end{array}
\]

\textbf{Objetivo:} Estimar \(\alpha\) y \(\beta\) con base en observaciones de \(x\) y \(y\).\\
Esta desviación de un \(y_i\) alrededor de su valor esperado se expresa como:

\[
\begin{aligned}
u_i &= y_i - E(y|x_i) \\
y_i &= E(y|x_i) + u_i
\end{aligned}
\]

Donde \(u_i\) es el término de error estocástico, que representa todas las variables omitidas que puedan afectar \(y\), pero no están incluidas en el modelo de regresión.

\[
\begin{aligned}
y_i = \alpha + \beta x_i + u_i \quad \therefore \quad E(y|x_i) = \alpha + \beta x_i \\
\therefore \quad y_i = \alpha + \beta x_i + u_i \rightarrow \text{Ecuación de regresión}
\end{aligned}
\]

Adicionalmente, si se toma \(E(\cdot)\) a \(y_i\), tenemos que:
\[
\begin{aligned}
y_i &= E(y|x_i) + u_i \\
E(y|x_i) &= E[E(y|x_i)] + E(u_i|x_i) \\
&= E(y|x_i) + E(u_i|x_i) \quad \therefore \\
E(u_i|x_i) &= E(y|x_i) - E(y|x_i) \quad \therefore \\
E(u_i|x_i) &= 0
\end{aligned}
\]

\begin{center}\rule{0.5\linewidth}{0.5pt}\end{center}

\hypertarget{funciuxf3n-de-regresiuxf3n-muestral}{%
\section{Función de regresión muestral}\label{funciuxf3n-de-regresiuxf3n-muestral}}

A diferencia del caso anterior, en realidad se trabaja con muestras. Se estima la FRP con base en información \textbf{muestral.}
Suponer que se extrae una muestra d ela población de 60 familias donde se estudia el gasto en consumo y el ingreso familiar.
Muestra aleatoria de 10 observaciones.

{[}IMAGEN{]}

La función de regresión muestral (FRM) puede escribirse como:
\[
\hat y_i = \hat \alpha + \hat \beta x_i 
\]
Donde:

\begin{itemize}
\tightlist
\item
  \(\hat y_i \equiv\) \(y\) estimada
\item
  \(\hat \alpha_i \equiv\) estimador de \(\alpha\)
\item
  \(\hat \beta_i \equiv\) estimador de \(\beta\)
\end{itemize}

Se estiman los parámetros poblacionales a partir de información muestral.

\[
\begin{array}{ccc}
\text{Estimador} & \rightarrow & \text{Fórmula} \\
\text{Estimado} & \rightarrow & \text{Valor numérico}
\end{array}
\]

FRM (forma estocástica): \(y_i = \hat \alpha + \hat \beta x_i + \hat u_i\). Por lo tanto, \(\hat u_i\) es el residual muestral (estimado de \(u_i\)).

\begin{center}\rule{0.5\linewidth}{0.5pt}\end{center}

\hypertarget{objetivo-del-anuxe1lisis-de-regresiuxf3n}{%
\section{Objetivo del análisis de regresión}\label{objetivo-del-anuxe1lisis-de-regresiuxf3n}}

Estimar FRP, \(y_i = \alpha + \beta x_i + u_i\), a partir de estimar FRM, \(y_i = \hat \alpha + \hat \beta x_i + \hat u_i\).

{[}IMAGEN{]}

\hypertarget{conceptos-estaduxedsticos}{%
\section{Conceptos estadísticos}\label{conceptos-estaduxedsticos}}

\hypertarget{sumatoria-y-multiplicatoria}{%
\subsection{Sumatoria y multiplicatoria}\label{sumatoria-y-multiplicatoria}}

\[
\sum_{i=1}^n x_i = x_i + x_2 + \ldots + x_n
\]
Propiedades del operador de sumatoria \((\sum)\):

\begin{enumerate}
\def\labelenumi{\arabic{enumi}.}
\tightlist
\item
  \(\sum_{i=k}^n = nk \quad \therefore \quad k \equiv \text{Constante}\).
\item
  \(\sum_{i=1}^n k x_i = k\sum_{i=1}^nx_i \quad \therefore \quad k \equiv \text{Constante}\).
\item
  \(\sum_{i=1}^n (a+bx_i) = \sum_{i=1}^n a + \sum_{i=1}^n bx_i = na + b \sum_{i=1}^n x_i \quad \forall \quad a,b \equiv \text{Constante}\).
\item
  \(\sum_{i=1}^n (x_i + y_i) = \sum_{i=1}^nx_i + \sum_{i=1}^n y_i\).
\end{enumerate}

Se pueden tener sumatorias múltiples:
\[
\begin{aligned}
\sum_{i=1}^n\sum_{j=1}^m &= \sum_{i=1}^n (x_{i1} + x_{i2} + \ldots + x_{im}) \\
&= (x_{11} + x_{12} + \ldots + x_{1m}) + (x_{21} + x_{22} + \ldots + x_{2m})  + \ldots + (x_{n1} + x_{n2} + \ldots + x_{nm})
\end{aligned}
\]
Propiedades del operador de sumatoria \((\sum \sum)\):

\begin{enumerate}
\def\labelenumi{\arabic{enumi}.}
\tightlist
\item
  \(\sum_{i=1}^n\sum_{j=1}^m x_{ij} = \sum_{j=1}^m\sum_{i=1}^n \quad \rightarrow \quad \text{Intercambiable}\)
\item
  \(\sum_{i=1}^n\sum_{j=1}^m x_i y_j = \sum_{i=1}^n x_i \sum_{j=1}^m y_j\)
\item
  \(\sum_{i=1}^n\sum_{j=1}^m (x_{ij} + y_{ij}) = \sum_{i=1}^n\sum_{j=1}^m x_{ij} + \sum_{i=1}^n\sum_{j=1}^m y_{ij}\)
\end{enumerate}

Adicionalmente, el operador multiplicatoria:

\[
\Pi_{i=1}^n x_i = x_1\cdot x_2 \cdot \ldots \cdot x_n
\]

\hypertarget{valor-esperado}{%
\subsection{Valor esperado}\label{valor-esperado}}

El valor esperado de una variable aleatoria discreta se define como:
\[
E(x) = \sum_x xf(x)
\]
\[
\begin{array}{ccc}
\therefore \quad x & \rightarrow & \text{Valores de la variable aleatoria discreta} \\
f(x) & \rightarrow & \text{FDP (discreta) de x}
\end{array}
\]

En términos poblacionales:

\[
E(x) = \mu_x \quad \rightarrow \quad \text{La media de la variable aleatoria discreta}
\]

Propiedades del valor esperado \(E(\cdot)\):

\begin{enumerate}
\def\labelenumi{\arabic{enumi}.}
\tightlist
\item
  \(E(b) = b \quad \therefore \quad b \equiv \text{Constante}\)
\item
  \(E(ax + b) = E(ax) + E(b) = aE(x) + b \quad \forall \quad a,b\equiv \text{Constantes}\)
\item
  \(E(xy) = E(x)E(y) \quad \forall \quad x,y \quad \rightarrow \quad \text{Variables aleatorias independientes}\)
\end{enumerate}

\hypertarget{varianza}{%
\subsection{Varianza}\label{varianza}}

Sea \(x\) una variable aleatoria y \(E(\cdot) = \mu\). La dispersión de valores de \(x\) alrededor de la media (valor esperado) es:

\[
var(x) = \sigma_x^2 = E(x-\mu)^2
\]

La desviación estándar de \(x\) es:

\[
\text{Desv. est. }(x) = \sigma_x = [E(x-\mu)]^{1/2}
\]

La varianza y desviación estándar muestral se calcula como:

\[
\begin{array}{cc}
s_x^2 = \frac{\sum_{i=1}^n (x_i - \bar x)}{n}, & s_x = \sqrt{\frac{\sum_{i=1}^n (x_i - \bar x)}{n}}
\end{array}
\]

Propiedades de la varianza:

\begin{enumerate}
\def\labelenumi{\arabic{enumi}.}
\tightlist
\item
  \(E(x-\mu) = E(x^2) - \mu^2 \quad \rightarrow \quad \text{Nota: Al desarrollar el cuadrado se simplifica.}\)
\item
  \(var(c)=0 \quad \because \quad c \equiv \text{Constante}\)
\item
  \(var(ax + b) = var(ax) + var(b) = a^2 var(x) + 0 = a^2 var(x)\)
\item
  \(var(x+y) = var(x) + var(y) \quad \because \quad x,y \text{ son variables aleatorias independientes}\)
\item
  \(var(ax + by) = var(ax) + var(by) \quad \because \quad x,y \text{ son variables aleatorias independientes}\)
\item
  \(var(x \pm y) = var(x) + var(y) \pm 2cov(x,y) \quad \because \quad x,y \text{ son variables aleatorias correlacionadas}\)
\end{enumerate}

\hypertarget{covarianza}{%
\subsection{Covarianza}\label{covarianza}}

Sean \(x\) y \(y\) dos variables aleatorias con medias \(\mu_x\) y \(\mu_y\), respectivamente. La covarianza se define como:

\[
cov(x,y) = E[(x-\mu_x)(y-\mu_y)] = \sigma_{xy}
\]

La covarianza muestral se calcula como:

\[
cov(x,y) = \frac{\sum_x \sum_y (x_i - \bar x)(y_i - \bar y)}{n} = s_{xy}
\]

Propiedades de la covarianza:

\begin{enumerate}
\def\labelenumi{\arabic{enumi}.}
\tightlist
\item
  \(cov(x,y) = E(xy) - \mu_x \mu_y = \mu_x \mu_y - \mu_x \mu_y = 0 \quad \because \quad x,y \text{ son independientes}\)
\item
  \(cov(a+bx, c+dy) = E\left[ \left[(a+bx) - E(a+bx)\right]\left[(c+dy) - E(c+dy)\right] \right] \\ = E\left[ \left[ bx - bE(x) \right]\left[ dy - dE(y) \right] \right] \\ = bd \, E\left[(x-\mu_{x})(y-\mu_{y})\right] \\ = bd \, cov(x, y)\)
\item
  \(cov(c,x) = 0 \quad \because \quad c \equiv \text{Constante}\)
\item
  \(cov(c,d) = 0 \quad \because \quad c,d \text{ son constantes}\)
\end{enumerate}

\hypertarget{coeficiente-de-correlaciuxf3n}{%
\subsection{Coeficiente de correlación}\label{coeficiente-de-correlaciuxf3n}}

El coeficiente de correlación (poblacional), se define como:

\[
corr(x,y) = \rho = \frac{cov(x,y)}{\sqrt{var(x)var(y)}} = \frac{\sigma_{x,y}}{\sigma_x \sigma_y}
\]
\(\rho\) es una medida de asociación lineal entre 2 variables \(\rightarrow \quad -1 \leq \rho \leq 1\).
El coeficiente de correlación (muestral se tiene:)

\[
corr(x,y) = \rho =  \frac{\sum_x \sum_y (x_i - \bar x)(y_i - \bar y)}{\sqrt{\sum_x (x_i - \bar x)^2} \sqrt{\sum_y(y_i - \bar y)^2}}
\]

\hypertarget{modelo-de-regresiuxf3n-lineal-simple}{%
\chapter{Modelo de regresión lineal simple}\label{modelo-de-regresiuxf3n-lineal-simple}}

Suponer una FRP de dos variables:

\[
y_i = \alpha + \beta x_i + u_i
\]

Dado que la FRP no es observable, se estima la FRM:

\[
\begin{array}{ccc}
y_i = \hat \alpha + \hat \beta x_i + \hat u_i & \rightarrow & \hat y_i = \hat \alpha + \hat \beta x_i \, \therefore \\
y_i = \hat y_i + \hat u_i & & \because \, \hat y_i \equiv \text{Valor estimado de } \hat y_i
\end{array}
\]

¿Cómo se estima la FRM?

\[
\begin{aligned}
y_i &= \hat y_i + \hat u_i \, \therefore \\
\hat u_i &= y_i - \hat y_i \\
&= y_i - (\hat \alpha + \hat \beta x_i) \\
&= y_i - \hat \alpha - \hat \beta x_i \, \rightarrow \hat u_i \equiv \text{Residuos}
\end{aligned}
\]

\hypertarget{muxednimos-cuadrados-ordinarios}{%
\section{Mínimos cuadrados ordinarios}\label{muxednimos-cuadrados-ordinarios}}

\textbf{Objetivo:} Seleccionar la FRM de tal forma que la suma de los residuos al cuadrado sea la menor posible \(\rightarrow \, \sum \hat u_i^2 = \sum (y_i - \hat y_i)^2\).

{[}IMAGEN{]}

\textbf{Criterio de mínimos cuadrados:}

\[
\begin{aligned}
\sum \hat u_i^2 &= \sum (y_i - \hat y_i)^2 \, \rightarrow \hat y_i = \hat \alpha + \hat \beta x_i \, \therefore \\
&= \sum(y_i - \hat \alpha - \hat \beta x_i)^2 \, \therefore \\
 \min_{\hat \alpha, \hat \beta} \sum \hat u_i^2 &= \sum(y_i - \hat \alpha - \hat \beta x_i)^2
\end{aligned}
\]

CPO (Condiciones de primer orden):

\[
\begin{aligned}
\frac{ \partial \sum \hat{u}_{i} }{ \hat{\partial} \alpha } &= 2 \sum (y_{i} - \hat{\alpha } - \hat{\beta}x_{i})(-1) = 0  \\
&= -2 \sum (y_{i} - \hat{\alpha } - \hat{\beta}x_{i}) = 0 \\
&= \sum (y_{i} - \hat{\alpha } - \hat{\beta}x_{i}) = 0 \\
\sum y_{i} &= n \hat{\alpha } + \hat{\beta}\sum x_{i} \, \rightarrow \text{Ecuación normal (1)}
\end{aligned}
\]

\[
\begin{aligned}
\frac{ \partial \sum \hat{u}_{i} }{ \hat{\partial} \beta } &= 2 \sum (y_{i} - \hat{\alpha } - \hat{\beta}x_{i})(-x_{i}) = 0  \\
&= -2 \sum (y_{i} - \hat{\alpha } - \hat{\beta}x_{i})(x_{i}) = 0 \\
&= \sum (y_{i}x_{i} - \hat{\alpha } x_{i} - \hat{\beta}x_{i}^2) = 0 \\
&= \sum y_{i}x_{i} - \hat{\alpha }\sum x_{i} - \hat{\beta}\sum x_{i}^2 = 0 \\
\sum y_{i}x_{i} &= \hat{\alpha }\sum x_{i} + \hat{\beta}\sum x_{i}^2 \, \rightarrow \text{Ecuación normal (2)}
\end{aligned}
\]

Retomemos las ecuaciones normales \((1)\) y \((2)\) y resolvemos para \(\hat \alpha\) y \(\hat \beta\):

\[
\left .  
\begin{aligned}  
\sum y_{i} &= n \hat{\alpha } + \hat{\beta}\sum x_{i} \\
\sum y_{i}x_{i} &= \hat{\alpha }\sum x_{i} + \hat{\beta}\sum x_{i}^2 
\end{aligned} 
\right\}  
\quad \mathrm{Ax = d}
\]

\[
\mathrm{A} = 
\begin{bmatrix}
n & \sum x_{i} \\
\sum x_{i} & \sum x_{i}^2
\end{bmatrix}_{2 \times 2}
,
\quad 
\mathrm{x} =
\begin{bmatrix}
\hat{\alpha} \\
\hat{ \beta}
\end{bmatrix}_{2 \times 1}
,
\quad
\mathrm{d} = 
\begin{bmatrix}
\sum y_{i}  \\
\sum x_{i} y_{i}
\end{bmatrix}_{2 \times 1}
\]

  \bibliography{book.bib,packages.bib}

\end{document}
